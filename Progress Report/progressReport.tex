\documentclass[10pt,draftclsnofoot,onecolumn]{IEEEtran}
\usepackage{graphicx}
\usepackage{textcomp}

\graphicspath{ {} }

%TODO need title page from prev documents/ other packages.
% Page layout (geometry)
\setlength\voffset{-1in}
\setlength\hoffset{-1in}
\setlength\topmargin{0.5in}
\setlength\oddsidemargin{.75in}
\setlength\evensidemargin{.75in}
\setlength\textheight{8.278in}
\setlength\textwidth{6.5in}
\setlength\footskip{0.561in}
\setlength\headheight{0.5in}
\setlength\headsep{0.461in}
\begin{document}
\pagenumbering{gobble}
\title{Fall Progress Report}
\author{Group 20: Shawn Cross, Griffin Gonsalves, Paul Kwak}
\maketitle
\hspace*{\fill}\IEEEauthorblockA{Capstone - Fall 2016}\hspace*{\fill}
\vspace{2cm}
\begin{abstract}
This document expands upon Group 20's progress with the project over the course of the term, and includes a project summary, details of project related problems with changes and solutions, as well as a week-by-week breakdown of the progress of the design phase. 
\end{abstract}
\IEEEpeerreviewmaketitle

\newpage
\pagenumbering{arabic}
%TODO Three necessary items -- report (as archive of all needed files to build as well as a PDF), slides (also as archive if necessary), and presentation.
%All text based items should be in your git space, on the master branch.
%All three items must also be in your ENGR web space
%Email all three links to both Kirsten, Kevin, and your TA. The email should have the full text of the links, one per line, and no additional information. Use plain text emails, if you would.

%Images can be added to explain points as well.
\section{Project Recap}
%Purposes and Goals
Our project seeks to improve and build upon vrok.it increasing the usability while simultaneously integrating a couple more functionality using Forge APIs. The end product will be capable of uploading CAD files via local user files or through Data Management’s A360, displaying the files as models through the File Viewer, file conversion with the Model Derivative, viewing the site on a mobile device via QR scanner and then viewing the site, specifically models, in VR with Google Cardboard.

Our team is motivated to provide these powerful features and to gain experience developing a larger scale project. While the foundation is now laid out, we are excited to add more ideas and additional features to our project. For instance, in order to provide support for additional VR devices, we want to ensure the project is useful on a low cost scenario by utilizing Google's Cardboard headset.

\section{Current Progress}
As this is the end of the planning phase, all core documentation has been taken care of. All we have to do now is to begin building it! We have all the documentation from Forge on how to integrate the various API and have the source code for Vrok-it. Once we are able to implement the core site features from Vrok-it, we will be able to introduce our new components and features shortly after focusing in on our development plan.

Moving forward, each of us know the components we are handling and we are developing strategies that will make sure we stay on the same page and that development is not hindered. This term, we have communicated and collaborated through Slack and Facebook messenger with success. We are sure we will continue to use these services for organization and planning, and certainly during development. During times of heavier development, additional in-person meetings or phone calls between team members should help keep everyone up to speed.


\section{Problems and Solutions}
Client meetings were not necessarily a problem overall. However, the client pulled in some additional people leading to extra time meeting being taken to get everyone on the same page. Furthermore, with the busy schedule of the Autodesk members involved with a large education event. We had a couple weeks where communication was sparse. As a result, it took a while before our requirements document was refined properly, several days after our original deadline. In the final week of class, we discovered after it went through grading that we missed adding grading metrics to the requirements document. However, this is only a small setback that we will take care of before the next phase begins. 


It was also around the same time of this two-week period with low communication that one of our members fell ill and also communication was at a low from them as well. This lead to them having low contribution for that time period. This was resolved with a meeting where these problems were brought up and things were talked out.

\section{Retrospective}

\begin{center}
\begin{tabular}{ |p{0.3\linewidth}|p{0.3\linewidth}|p{0.3\linewidth}| } 
 \hline
 \textbf{Positives} & Deltas & Actions \\ 
\hline
After Autodesk University was over our client has become much more available to answer our questions and concerns regarding the project & Planning for future events with the team and making sure there aren't any major conflicts. & The team will communicate questions through emails, and scheduling conflicts will be resolved during client meetings. \\
\hline
  & Better team communication. & This could be accomplished by setting up an in person weekly meeting that happens at the same time every week. \\
\hline 
  & Addressing problems or concerns regrading assignments or documentation. & If everyone in the group is having the same problem and we are not able to come up with a solution we need to ask our TA our one of the professors if they have the answer to the problem we are having. \\ 
 \hline
\end{tabular}
\end{center}


\section{Weekly summaries}
\subsection{Week 3}
During this week we were able meet with our client Patti Vrobel and begin to figure out what the we were going to be doing for our project. Patti wanted us to come up with ideas they would make use of the Autodesk Forge APIs. Later that week we had another meeting with Patti and Jim Quanci to figure out if we would be capable of completing the ideas that we had come up with in the time that we had. Jim also gave us a suggestion on what we could do as our project and all of us agreed that the project sounded like something we wanted to do. The biggest issue we had ran into up to this point was trying to get a good idea of what it was exactly that we were going to be doing for our project.   

\subsection{Week 4}
This week we were able to really begin working on our problem statement assignment and start figuring out how exactly we planned on expanding the vrok.it project that we would be starting from. We also had another conference call with Patti and were able to get many of the questions we had about the project cleared up. We were also able to get in contact with the engineer behind the vrok.it project, Kean, and he provided us with more information about what he had intended the project to be. The problem that we were having this week was that we didn't feel like we had an adequate amount of information about what we were actually going to be doing on the project to write a complete problem statement. 

\subsection{Week 5}
During Week 5 we were able to meet up and have a conference call with our client and Kean Walmsley on Monday and Tuesday. During these meetings we were able to complete the project statement and begin preliminary work on the requirements document. It was challenging for us to try to narrow the scope and find more specific goals for the project, and having our group meeting with Patti was very helpful in that regard. At this point we decided to focus more on the user experience end, focusing on usable features, and increasing performance when possible rather than attempting to tackle a challenging optimization. With the fog cleared, we were able to complete the statement document rapidly. In the coming days we were able to use our client meeting to drive our requirements document, and laid out some of the fundamental ideas, such as the viewer and VR content.

\subsection{Week 6}
Week 6 we had hoped to be able to meet with our client to go over some of the details of our project that we were still unsure about, however we were unable to do so. We had also talked with our TA Nels about joining us on our next conference call with the client to help us figure out if what the client was wanting us to do was actually possible in the amount of time that we have. At this point the biggest problem we had was that we were not able to communicate with our client very well because they were in the process of getting ready for a big conference that they host every year. This was challenging for us because we had a lot of questions about the project and the requirements but we were unable to get them answered.

\subsection{Week 7}
During this week we were finally able to get a draft of our requirements document done and get feedback from Nels. His biggest suggestion to us was to be more detailed in the descriptions of our specific requirements. We also had come up ways in which we could use more of the Autodesk APIs, so we also wanted to add those to the final version of the document that we would be turning in as well. We were able to get a hold of Patti this week and set up a conference call with everyone for some time during the next week. We were all looking forward to finally getting to talk to everyone and hopefully get all of the questions at this point answered. The technology review assignment also began during this week. The biggest problems we had during this week were trying to finish the revision of our requirements document and do our individual tech reviews at the same time. Also Paul was very sick this week and was unable to help as much as he normally would. 

\subsection{Week 8}
This week started with the tech reviews being turned in. As a group we all had the same feeling that while the individual reviews we did weren't bad they could have definitely been better if we had a little more time. Although a bit time consuming we did find the tech reviews to be useful in figuring out exactly how we planned on completing each of our individual tasks. We were also set up to have a conference call with Patti and the other people involved with the project to close out the week. Nels was also going to join us for this call to help us determine if some of the performance stuff that Jim had suggested we try to do was something we could actually accomplish. Overall this was a very productive week for our group and we all felt like we were finally starting to get caught back up this week. We did not run into to many problems during this week other than some minor things while trying to do our individual tech reviews. 

\subsection{Week 9}
This week was probably the least productive week of the term for our group. This was the week of thanksgiving so we were all doing some traveling and trying to spend time with our families. We did however try to start the design document that was due the following Friday. This is where we ran into one of the biggest problems we had trying to do this assignment and that was understanding the information in the IEEE document that we were using for this assignment. As a group we were able to get a start on the document but did not get nearly as far on it as we had hoped to over the long weekend. We did manage to have a conference call with everyone to start the and got a lot of the questions that we still had answered. This allowed us to finish the final draft of our requirements document and get the document turned into Nels and Kirsten.  

\subsection{Week 10}
You could definitely tell that this term was coming to an end this week with everyone in our group running out of steam. We all had a ton of assignments and projects due this week and at the beginning of the next week and were all getting very little sleep trying to get it all done. In this class the design document was due on Friday by noon and we all really wanted to have it done by that deadline. We were still having problems trying to figure out how to use the IEEE 1016 doc at the beginning  of this week but we were able to talk to both Kirsten and Nels and really get a good idea of how to actually use the information in the 1016 to create a good design document. After that we were really able to get going on the document and we all spent a lot of time during this week adding to and revising the document and in the end we thought that the document had turned out to be really well done with a lot of good information for both us and our client. 

\end{document}

\documentclass[10pt,draftclsnofoot,onecolumn]{IEEEtran}
\usepackage{graphicx}
\usepackage{textcomp}

\graphicspath{ {} }

%TODO need title page from prev documents/ other packages.
\begin{document}
\pagenumbering{gobble}
\title{Progress Report}
\author{Shawn Cross, Griffin Gonsalves, Paul Kwak}
\maketitle
\hspace*{\fill}\IEEEauthorblockA{Capstone - Fall 2016}\hspace*{\fill}
\vspace{2cm}
\begin{abstract}

\end{abstract}
\IEEEpeerreviewmaketitle

\newpage
\pagenumbering{arabic}
%TODO Three necessary items -- report (as archive of all needed files to build as well as a PDF), slides (also as archive if necessary), and presentation.
%All text based items should be in your git space, on the master branch.
%All three items must also be in your ENGR web space
%Email all three links to both Kirsten, Kevin, and your TA. The email should have the full text of the links, one per line, and no additional information. Use plain text emails, if you would.

%Images can be added to explain points as well.
\section{Project Recap}
%Purposes and Goals
Our project seeks to improve and build upon vrok.it increasing the usability while simultaneously integrating a couple more functionality using Forge APIs. The end product will be capable of uploading CAD files via local user files or through Data Management’s A360, displaying the files as models through the File Viewer, file conversion with the Model Derivative, viewing the site on a mobile device via QR scanner and then viewing the site, specifically models, in VR with Google Cardboard.

Our team is motivated to provide these powerful features and to gain experience developing a larger scale project. While the foundation is now laid out, we are excited to add more ideas and additional features to our project. For instance, in order to provide support for additional VR devices, we want to ensure the project is useful on a low cost scenario by utilizing Google's Cardboard headset.

\section{Current Progress}
As this is the end of the planning phase, all core documentation has been taken care of. All we have to do now is to begin building it! We have all the documentation from Forge on how to integrate the various API and have the source code for Vrok.it.

Strategy for Development:
Moving forward, each of us know the components we are handling and we need a strategy that will make sure we stay on the same page. This term, we have communicated and collaborated through Slack and Facebook messenger with success. I'm sure we will continue to use these services for organization and planning, and certainly during development. During times of heavier development, additional in-person meetings or phone calls between team members should help keep everyone up to speed.


\section{Problems and Solutions}
Client meetings weren’t necessarily a problem overall. However, the client pulled in some additional people leading to extra time meeting being taken to get everyone on the same page. Furthermore, with the busy schedule of the Autodesk members involved with a large education event. We had a couple weeks where communication was sparse. As a result, it took a while before our requirements document was refined properly, several days after our original deadline. In the final week of class, we discovered after it went through grading that we missed adding grading metrics to the requirements document. However, this is only a small setback that we will take care of before the next phase begins. 


It was also around the same time of this two week period with low communication that one of our members fell ill and also communication was at a low from them as well. This lead to them having low contribution for that time period. This was resolved with a meeting where these problems were brought up and things were talked out.

\section{Retrospective}

\begin{center}
\begin{tabular}{ |c|c|c| } 
 \hline
 cell1 & cell2 & cell3 \\ 
 cell4 & cell5 & cell6 \\ 
 cell7 & cell8 & cell9 \\ 
 \hline
\end{tabular}
\end{center}


\section{Weekly summaries}
\subsection{Week 3}
	During this week we were able meet with our client Patti Vrobel and begin to figure out what the we were going to be doing for our project. Patti Wanted us to come up with Ideas the would make use of the Autodesk Forge APIs. Later that week we had another meeting with Patti and Jim Quanci to figure out if we would be capable of completing the ideas that we had come up with in the time that we had. Jim also gave us a suguestion on what we could do as our project and all of us agreed that the project sounded like something we wanted to do. The biggest issue we had ran into up to this point was trying to get a good idea of what it was exactly that we were going to be doing for our project.   
\subsubsection{Shawn}
\subsubsection{Griffin}
\subsubsection{Paul}

\subsection{Week 4}
	This week we were able to really begin working on our problem statement assignment and start figuring out how exacty we planned on expanding the vrok.it project that we would be starting from. We also had another conference call with Patti and were able to get many of the questions we had about the project cleared up. We were also able to get in contact with the engineer behind the vrok.it project, Kean, and he provided us with more information about what he had inteded the project to be. The problem that we were having this week was that we didn't feel like we had an adequate amount of information about what we were actually going to be doing on the project to write a complete problem statement. 
\subsubsection{Shawn}
\subsubsection{Griffin}
\subsubsection{Paul}

\subsection{Week 5}
During week 5 we were able to meet up and have a conference call with our client on monday and tuesday. During these meetings we were able to complete the project statement and begin preliminary work on the requirements document. It was challenging for us to try to narrow the scope and find more specific goals for the project, and having our group meeting with Patti was very helpful in that regard. At this point we decided to focus more on the user experience end, focusing on usable features, and increasing performance when possible rather than attempting to tackle a challenging optimization. With the fog cleared, we were able to complete the statement document rapidly. In the coming days we were able to use our client meeting to drive our requirements document, and laid out some of the fundamental ideas, such as the viewer and VR content.
\subsubsection{Shawn}
\subsubsection{Griffin}
\subsubsection{Paul}

\subsection{Week 6}
\subsubsection{Shawn}
\subsubsection{Griffin}
\subsubsection{Paul}

\subsection{Week 7}
\subsubsection{Shawn}
\subsubsection{Griffin}
\subsubsection{Paul}

\subsection{Week 8}
\subsubsection{Shawn}
\subsubsection{Griffin}
\subsubsection{Paul}

\subsection{Week 9}
\subsubsection{Shawn}
\subsubsection{Griffin}
\subsubsection{Paul}

\subsection{Week 10}
\subsubsection{Shawn}
\subsubsection{Griffin}
\subsubsection{Paul}



\end{document}

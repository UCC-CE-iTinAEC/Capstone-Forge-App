\documentclass[letterpaper,10pt,draftclsnofoot,onecolumn, titlepage]{IEEEtran}

\usepackage{color}
\usepackage{url}
\usepackage{array}
\usepackage{listings}




%\usepackage{balance}
%\usepackage[TABBOTCAP, tight]{subfigure}
\usepackage{enumitem}
\usepackage{pstricks, pst-node}

\usepackage{geometry}
\geometry{textheight=8.5in, textwidth=6in}




\newcommand{\toc}{\tableofcontents}

\usepackage{hyperref}\usepackage{digsig}
%########################################
%CS 461 CAPSTONE 
%GROUP 20 - Griffin Gonsalves, Paul Kwak, Shawn Cross
%FALL 2016
%########################################
\begin{document}

\begin{titlepage}
\centering
{\huge Forge API Project Statement\par}
\vfill
{\LARGE Griffin Gonsalves, Shawn Cross, Paul Kwak\par}
{\vspace{2mm}}
{\large CS444 - Senior Capstone Fall Term. \par}
{\vspace{10mm}}
%{\large Abstract here?\par}
\end{titlepage}

%normal doc begins
\section{Abstract}
{Group 20’s project branches from an Autodesk prototype project called Vrok-It, which is a simple web-based 3D model viewer and mobile virtual reality (VR) explorer. Group 20’s project will expand upon its ability to display uploaded 3D models in browser and in VR, and improve its accessibility. Conventionally, viewing 3D models in VR is a challenge if you have model files on many devices, or have a headset that only works in conjunction with a smartphone. Group 20’s project aims to do this by utilizing a web-based software that uses the features of the Autodesk Forge API. The project will also be expanded with new ideas and stretch goals as the project is developed. \par}

\section{Problem Definition}
{With a lower cost of entry to VR, many more users are now able to take advantage of these technologies. For example, several Android smartphones are now supporting VR as well, potentially introducing millions of users to this new front. We see an opportunity to enable these users to share simple 3D projects in a way that is both new and familiar, and allows for modern VR technologies to be applied. Engineers, 3D artists, students, or anyone who works with 3D models and has a VR solution is included in our potential users. This project would appeal to a wide audience as it will be made available through a website at no cost. 

However, most of the CAD software that is currently available can be prohibitively expensive and most people that are not experienced with using CAD software find it very difficult to use. This barrier prevents an average user from quickly viewing a 3D project. Additionally, there is also not a lot of software available that allows users to easily and affordably go between viewing their CAD files in a 3D model viewer and on a VR device. Using the website and a VR technology, users could take any CAD projects they have, instantly upload them into a 3D model viewer as well as have it viewable on their 3D device. 

However, each different VR enabled device can operate differently depending on the hardware that it is using. With so many different devices it is hard to know exactly what a user would be able to display on their specific device. This generally means that certain devices that have less powerful hardware can have poor experiences when viewing objects in the viewer or with VR. Other performance limitations are influenced by larger and more complex models that are very detailed or have many of different parts can take a lot of effort to render in 3D. If a device cannot meet the requirement of the VR solution, then the experience of the user would be negative. Most users do not want to have to worry about this and just want the software to make it work for them on the device they have. 
 
 
\par}

\section{Proposed Solution}
{If approved, we will be working from the Vrok-It project that was started by Kean Walmsley, Lars Schneider, Oleg Dedkow. By forking Vrok-It, we will implement our own changes to make the site more accessible to users and to their hardware solutions. To make the project more accessible we would need to create software to enable Vrok-It to detect VR devices, and make changes to the viewing experience based off of this. Using this software, we would also like to ensure that if a model is too large or complex, the program will alert the user.

Another proposed method for improving performance is to develop software to determine if the model that is being uploaded has many internal components that cannot actually be seen by the user in the VR environment. If this is the case then the software would then remove these components from the model prior to being viewed in VR, increasing the ability of the device to display the model. We would like to develop these adjustments to remove many roadblocks from viewing a 3D file in VR. By doing this, we hope to allow users to become more interactive with their work and further drive creativity.

Beyond our initial goal of improving the experience, we will also be developing additional goals based on our progress, and feedback from our client. 

\par}

\section{Performance Metrics}
{Initially, we hope to take a deeper look into Vrok.it, and focus on improving usability on more devices and hardware. Our goal in mind for this objective is to implement a solution to improve performance on lower-spec devices. When viewing models there should be a smooth framerate in order to have a good viewing experience in VR, so maximizing our performance is essential on those devices. The finished project will be able to make suggestions on what the users object quality should be based on the device that they are currently using. We hope that we will be able to demonstrate the project using a VR or augmented reality device, and the project site. \par}

\newpage
\null
\vfill

\begin{flushleft}


	\begin{Form}
		\digsigfield{14cm}{3cm}{Sign Here} 	
		\rule{5in}{.4mm}\\
			Patti Vrobel\hspace{60ex}Date
	\end{Form}
		
	\vspace{1cm}	
	
	\rule{5in}{.4mm}\\
	Griffin Gonsalves\hspace{55ex}Date
		
	\vspace{1cm}	
	
	\rule{5in}{.4mm}\\
	Shawn Cross\hspace{59ex}Date
	
	\vspace{1cm}	
	
	\rule{5in}{.4mm}\\
	Paul Kwak\hspace{61ex}Date

\end{flushleft}
%bib stuff
%\bibliographystyle{IEEEtran}
%\bibliography{gonsalvg_writingAssign1_038}
\end{document}